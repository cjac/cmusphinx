\documentclass{article}
\title{Scalable Speech Recognition Using Distributed Annotators} 
\author{David Huggins-Daines}
\begin{document}
\maketitle

\begin{abstract}


\end{abstract}

\tableofcontents{}

\section{Introduction}
\label{sec:intro}

As Moore's Law continues to give us more and smaller transistors,
computers as we know them are undergoing a radical transformation.  We
are currently witnessing the nascent ubiquity of intelligent mobile
devices; for instance, the cellular phones of today are more capable
in terms of raw processor speed than the personal computers of a mere
10 years ago.  Many people are now rarely outside an arm's reach of
two or more fully programmable, multimedia capable and Internet
connected devices, be they smartphones, laptops, or conventional PCs
and workstations.  Wireless networks, too, have become ubiquitous, as
they provide a convenient way both to access Internet content and move
data between mobile devices.

In some respects, however, the growth curve of computational capacity
has begun to flatten.  The single-threaded performance of modern CPUs
has mostly stopped increasing, as clock speeds have levelled off and
gains in instruction-level parallelism are becoming increasingly hard
to come by.  In addition, as more and more interesting applications
rely on Internet access, network latency rather than processor speed
has become the deciding factor in whether a computer performs
adequately.

What do these trends mean for the future of automatic speech
recognition?  This thesis is an attempt to answer this question, by
proposing an approach to speech recognition based on the concept of
lightweight, independent {\em annotators} which collaborate, either
sequentially or in parallel, on a structured representation of speech.

\section{Automatic Speech Recognition}
\label{sec:asr}

\section{Thesis Statement}
\label{sec:thesis}




\bibliographystyle{apalike}
\bibliography{proposal}

\end{document}
